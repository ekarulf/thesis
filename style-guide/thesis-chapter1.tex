\chapter{Thesis Format}
\label{cpt:format}

The following guidelines offer you some degree of flexibility in formatting
your thesis. Options are summarized in Table~\ref{tab:options}.  Whatever
options you choose to use, you must use them consistently throughout the document.

\section{Margins}

Your \underline{printed output} must reflect a \underline{physically
measurable} left margin of at least 1.5 inches, with top, bottom, and right
margins measurable at 1 inch.  Some systems' settings produce varying results
when printing to different printers, so be sure to measure your output.
Remember, nothing (not even page numbers) should print in the margins.

\section{Page Numbers}

Number all pages in your thesis except the title page and the optional
copyright page which might follow the title page.  Number the ``front matter''
pages (i.e., the pages that come prior to the main body of text, prior to
chapter 1) with lowercase Roman numerals, centered immediately above the bottom
margin, and starting with the Roman numeral ``ii''.  Number the pages starting
with the first page of the first chapter with Arabic numerals, also centered
immediately above the bottom margin, and starting with numeral ``1''.

\section{Body Text}

Use a 10, 11, or 12-point Garamond, Times Roman or Times New Roman font for
your thesis text.  {\scriptsize (The MicroSoft WORD based ``template'' uses
Garamond throughout, and is recommended whenever possible.  The \LaTeX{}
version uses a high quality variation of the Times Roman font.  Whichever is
used, be consistent throughout your document..)}  Use 1.5 or double line
spacing for most body text.  Block quotes should be single spaced.  Use either
left justification with a ragged right edge, or full justification.  Paragraphs
may be set in a block style, with no indentation, or they may be indented up to
0.5 inch.  Skip a line between paragraphs.

\section{Titles and Headings}

Titles and headings may be left-justified or centered.  Capitalize the first
letter of the first word and the first letter of each subsequent major word in
a title or heading.  Do not capitalize articles, prepositions, and conjunctions
that are not the first word of a title or heading.  For example, do not
capitalize such words as the following: a, an, the, for, to, on, or.
Formatting specifications for particular types of headings and titles are
described below.  You may use a plain or bold version of the body text font for
all titles and headings.

\subsection{Chapter Titles}

Begin each chapter on a new page.  You may start the chapter title below the
top margin  (1.5 inches from the top edge of the page), or you may leave some
space and start the chapter title up to 3 inches from the top edge of the page.
You may use a font size of up to 36 points for the chapter title.  There are 2
options for formatting the chapter title:
\begin{itemize}
\item Type the word ``Chapter'' followed by the chapter number, skip a
  line, and type the chapter title on the following line; or
\item Type the chapter number followed by the chapter title, all on
  the same line.
\end{itemize}

\subsection{Section Headings}

You may use a font size of up to 24 points for the section headings.  Type the
chapter number and section number before the section title.

\subsection{Subsection Headings}

You may use a font size of up to 18 points for subsection headings.  Type the
chapter number, section number, and subsection number before the subsection
title.

\subsection{Headings for Divisions Smaller than Subsections}

Use unnumbered headings for divisions smaller than subsections.  You may use a
font size of up to 14 points.  Headings may be typed above or on the same line
as the sections they label.  You may use both styles within your thesis.

\paragraph{Run-in Headings}
To the left is an example of a run-in heading.  Notice that it is typed on the
same line as the section that it labels.  It may be used for divisions smaller
than subsections.

\begin{figure}[h]
\centering
\includegraphics{justafigure}
\caption{Just a Figure\label{fig:justafigure}}
\end{figure}

\section{Figures and Tables}

Figures and tables must be referenced in the text by number.  They must be
numbered consecutively throughout each chapter, with the chapter number
preceding each figure or table number.  For example, the third figure in
chapter 1 would be labeled Figure 1.3.  You may either:
\begin{itemize}
\item Maintain one numbering sequence for figures and another for
  tables, and label figures with the word ``Figure'' and tables with the
  word ``Table''; or
\item Label both figures and tables with the word ``Figure'' and
  maintain one numbering sequence.
\end{itemize}
Place figures and tables as close to their references in the text as
possible.  Place a figure number and title below each figure (or table
labeled as a figure).  Place a table number and title above each table
labeled as a table.  In figures and tables, avoid using color and
avoid text smaller than 10 points.  Do not let figures or tables spill
out into the margins.  Figure~\ref{fig:justafigure} is an example figure.

\section{Lists}

You may include lettered, numbered, or bulleted lists in your thesis.
Use consistent punctuation and capitalization throughout each list.
Lists may be indented.

\section{Footnotes and Endnotes}

You may use footnotes or endnotes for brief notes that are not appropriate for
the body of the text.  Use either footnotes or endnotes consistently throughout
your thesis.  Position footnotes in 10 point type just above the bottom margin
and page number.  Use a short horizontal rule to separate footnotes from the
text.  Position endnotes at the end of each chapter.  Type endnotes using the
same font size and justification as the body of the text.  Single space within
each footnote or endnote; double-space between footnotes or endnotes.
Footnotes and endnotes should be consecutively number.

\section{Quotations}

You must use quotation marks and parenthetical references to indicate words
that are not your own. Put quotation marks around short quotes.  Put long
quotes in separate single-spaced paragraphs, indented up to 1 inch from the
left margin (these are called block quotations).  Kate Turabian, editor of
official publications and dissertation secretary at the University of Chicago
for over 25 years, distinguishes short and long quotes as follows:

\begin{quote}
  Short, direct prose quotations should be incorporated into the text
  of the paper and enclosed in double quotation marks: ``One small step
  for man; one giant leap for mankind.'' But in general a prose
  quotation of two or more sentences which at the same time runs to
  four or more lines of text in a paper should be set off from the
  text and indented in its entirety\dots~\cite{Turabian}
\end{quote}

\section{Equations}

Equations may be set in-line with the text or numbered and placed in separate
paragraphs.  Use the same numbering style for equations as you would for
figures and tables.  Here is an example of an equation set in-line with a
paragraph: $E = mc^2$.  Here is an example equation placed in a separate
paragraph:
\begin{equation}
E = mc^2
\end{equation}
Equation numbering and formatting should follow the usual convention of
your discipline and be acceptable to your thesis committee.

\newpage
\begin{table}[h]
	\caption{Thesis Formatting Options\label{tab:options}}
	\vspace{0.125in}
	\centering
	\hyphenpenalty10000  % turn off hyphenation in this table.  It looks bad.
	
	\begin{tabular*}{\textwidth}{|c|@{\extracolsep{\fill}}c|}
		\hline
		\textbf{Thesis Element} & \textbf{Formatting Options} \\ \hline
		\textbf{title page font} & \small 12-point or 14-point Garamond, Times or Roman \\ \hline
		\textbf{table of contents chapter title} & \small bold or plain \\
			\textbf{font} & \\ \hline
		\textbf{first-level table of contents} & \small 0 to 0.5 inch \\
			\textbf{indentation} & \\ \hline
		\textbf{second-level table of contents} & \small 0 to 1.0 inch \\
			\textbf{indentation} & \\ \hline
		\textbf{body text font} & \small 10, 11, or 12-point Garamond, Times or Roman \\ \hline
		\textbf{body text line spacing} & \small 1.5 or 2 \\ \hline
		\textbf{body text justification} & \small left or full \\ \hline
		\textbf{paragraph indentation} & \small 0 to 0.5 inch \\ \hline
		\textbf{chapter title position} & \small 1.5 to 3 inches below top edge of page \\ \hline
		\textbf{chapter title style} & \small heading preceded by the word ``Chapter'' and \\
			& \small the chapter number or, heading preceded only \\
			& \small by the chapter number \\ \hline
		\textbf{chapter title} & \small 10-pt to 36-pt font, centered or left-justified, \\
			& \small plain or bold \\ \hline
		\textbf{section heading} & \small 10-pt to 24-pt font, centered or left-justified, \\
			& \small plain or bold \\ \hline
		\textbf{subsection heading} & \small 10-pt to 18-pt font, centered or left-justified, \\
			& \small plain or bold \\ \hline
		\textbf{unnumbered headings} & \small 10-pt to 14-pt font, centered or left-justified, \\
			& \small plain or bold \\ \hline
		\textbf{table labels} & \small label tables as ``Table'' or ``Figure'' \\ \hline
		\textbf{Parenthetical reference style} & \small author-date system, numbered, or another style \\
			& \small acceptable to your committee \\ \hline
		\textbf{Reference list style} & any style acceptable to your committee \\ \hline
	\end{tabular*}
\end{table}

%%% Local Variables: 
%%% mode: latex
%%% TeX-master: "thesis-main"
%%% End: 
