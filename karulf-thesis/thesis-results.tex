\chapter{User Study}
Technology is constantly changing the landscape of our society.  The introduction of new media like radio, television, and movies has permanently altered how we operate as a society and, in turn, how we as educate our young as a society.  The late twentieth century saw the wide spread proliferation and adoption of computers into the modern household.  Today, Internet communications and video games are entering our world as mainstream media.  As such, this new wave of technology impacts who we are on many levels. As a society, we must be aware of the risks associated with these newly emerging forms of media; as psychologists we must be alert to the influences of such an evolving medium on our culture, and as educators we must be mindful of how to properly utilize technology in the classroom.  The latter is the purpose of this paper: identifying critical issues of concern such as gender differences, socioeconomic obstacles, accessibility issues, and other limitations in technology which impact the integration of technology of kindergarten thru twelfth grade education. Fortunately, as the field of social psychology continues to evolve; new problems fuel new opportunities for further research into issues—and solutions—of technology integration in the classroom. 

\section{Testing Procedures}
An obvious problem with integrating technology into the classroom is it introduces an inherent bias; individuals skilled in the areas of math, science, and technology (MST) are at a distinct advantage.  In Voyles and Williams research on “Gender Differences in Attributions and Behavior in a Technology Classroom,” they highlight the National Science Foundation’s report that in the year 2000, women received only twenty one per cent of the undergraduate degrees in engineering and only twenty eight per cent of the bachelor’s degrees in computer science (Voyles \& Williams, 2004).  While this is an improvement from previous decades, this still represents a negative correlation between the female sex and careers in MST.  Social psychologists have studied this phenomenon and, while there are several prominent theories, there is not one specific or singular reason behind this large gender discrepancy.  
However, one popular theory proposed for such gender discrepancies is the application of Weiner’s Attribution Theory (Voyles \& Williams, 2004).  Simply put, students attribute their success or failure to a combination of four causes: ability, effort, luck, and task difficulty; these four attributions can be described with three descriptors or variables: internal versus external, stable versus unstable, and controllable versus uncontrollable.  (Voyles \& Williams, 2004). The attribution and descriptor definitions are shown below in Table 1 and reproduced from Voyles and Williams research (Voyles \& Williams, 2004, p. 235).  These attributes and self-perceptions are fascinating and provide unique insight into the dilemmas some young girls face when encountering an uncertain or new topic, such as learning how to use a computer and how to navigate on the Internet. This study could be further refined to modify the cognitions the girls use. 
Voyles and Williams also spotlight research done by Covington and Omelich, in 1979, identified as the Self Worth Theory (Voyles \& Williams, 2004). This theory describes males as having an external locus of control for failure and internal locus of control for success; conversely, the Self Worth Theory describes females as having an internal locus of control for failure and an external locus of control for success. Perhaps, this is one way to say girls tend to attribute their successes to external factors and attribute their failures to internal factors, while boys tend to attribute their successes to internal factors and attribute their failures to external factors.   
One could apply the aforementioned theory to a real life scenario. For example, if a girl failed an assigned computer task, she would attribute it to her own ability—or lack of ability—while a boy would attribute his performance to bad luck or an unreasonably difficult task.  Interestingly, studies show that as students grow older they seem to associate effort to ability, as if the amount of effort required is a reflection of the student’s ability itself. These misattributions of internal and external causes are actually an extension of a phenomenon known to psychologists as the Fundamental Attribution Error (FAE).  The FAE is defined as over-emphasizing internal causes and under-emphasizing external causes when judging another person’s behavior. Such misattributions, and their impact, are worthy of further research and study, specifically related to girls self image in MST. Girls must be empowered; we need to adopt effective tools to help young girls avoid cognitive distortions and to prevent further negative internalizations. Social psychology has the capacity for great impact in this area of gender differences and technology integration. 
In their 2004 experimental study, Voyles and Williams designed a technology rich learning enrichment summer camp to study these gender differences in ten thru twelve year-old students.  The study covered three consecutive summers, with each camp lasting for one week.  The first two summers the classes were divided by gender, while the third summer camp was comprised of co-educational classes.  The classes were split into cooperative teams that were designed to have students work together to create functioning robots.  Students and teachers were given questionnaires and interviews.  Additionally, groups were randomly videotaped for later processing by independent coders.  
In all three years, there were no significant statistical differences in the scores between the boys and the girls on the interviews or in the self-assessment surveys. (Voyles \& Williams, 2004)  However, the observed behavior on videotape suggested three major differences between the boys and girls.  First, the girls were more talkative in general; the girls both directed more questions to the teachers than their male counterparts, and they made more spontaneous attributions of success or failure in conversation with other students.  In a similar manner, girls were more likely to make failure comments primarily citing ability while boys were less likely to make failure comments, citing task difficulty when they did make failure comments.  Boys were more likely to make assured comments, confidently citing their own knowledge occasionally even if they were wrong (Voyles \& Williams, 2004).  
\section{On Screen Notifications}
In their 2004 experimental study, Voyles and Williams designed a technology rich learning enrichment summer camp to study these gender differences in ten thru twelve year-old students.  The study covered three consecutive summers, with each camp lasting for one week.  The first two summers the classes were divided by gender, while the third summer camp was comprised of co-educational classes.  The classes were split into cooperative teams that were designed to have students work together to create functioning robots.  Students and teachers were given questionnaires and interviews.  Additionally, groups were randomly videotaped for later processing by independent coders.  
In all three years, there were no significant statistical differences in the scores between the boys and the girls on the interviews or in the self-assessment surveys. (Voyles \& Williams, 2004)  However, the observed behavior on videotape suggested three major differences between the boys and girls.  First, the girls were more talkative in general; the girls both directed more questions to the teachers than their male counterparts, and they made more spontaneous attributions of success or failure in conversation with other students.  In a similar manner, girls were more likely to make failure comments primarily citing ability while boys were less likely to make failure comments, citing task difficulty when they did make failure comments.  Boys were more likely to make assured comments, confidently citing their own knowledge occasionally even if they were wrong (Voyles \& Williams, 2004).  
The final difference between girls and boys was the time spent “off task.” In general, it appeared that when girls were in a group of three, the group spent more time off task (Voyles \& Williams, 2004). The notable exceptions were groups comprised for one girl and two boys, the most “focused” group by a considerable margin. One wonders whether the girls’ performance was affected by the boy’s expectations and aforementioned gender biases in attribution. If so, perhaps possibilities exist to influence self-talk and self-identity among young girls. Clearly, more research is needed, but such studies identify the significant impact of gender differences on socialization and learning in technology education. 
Learning and socialization in technology education is not necessarily equitable, and research demonstrates that socioeconomic differences are pronounced.  In their large longitudinal study on the “digital divide” between the haves and have-nots, Judge, Puckett, and Bell provide illuminating research (2006).  The authors clearly state their case for understanding the role of socioeconomic status (SES) and technology education:
If one assumes that academic achievement is facilitated by access to computers at 
home and at school, the gap in access to computer technology is cause for   
concern. Digital equity is a social justice goal, ensuring that all students have 
access to information and communication technologies for learning, regardless of 
socioeconomic status (SES), disability, language, race, gender... (Judge et al, 2006 
p.52).  
     The study sample included public school children that attended kindergarten, first and third grades.  Although Judge and her colleagues reported, from their findings, that overall access to computers and technology is increasing, they did find differences in technology access between high-and low-poverty schools (Judge et al, 2006).  Furthermore, their research demonstrated that high and low poverty schools still do not have enough classroom computers, indicating the need still exists for all children to have access to computers, despite SES. Truly, discrepancies do exist in technological availability and comprehension at school and at home, based on SES. The teachers’ training and expertise also affects student learning (Judge et al, 2006).  For social psychologists, this is also a significant issue, as those in authority do have the power and status to strongly influence outcomes.  Further research might yield greater understanding of the lower SES schools, teachers’ attitudes and technology outcomes for those students. Another study finding of Judge, and her colleagues, suggests the availability and location of the computers—whether in the classroom, readily available, or in a computer laboratory—influences use and obviously the children’s expertise in using the computers. Children will more readily use that which is easily accessible. Some could view such dynamics as environmental manipulation that precludes lower SES students from achieving their academic goals. Although this study didn’t research the specific issue of personal comfort and ease with using the computer, it could be theorized that increased familiarity would assist students in using the computer as a resource more readily and perhaps it could positively impact academic accomplishments. However, academic gains were not found with increased use of computers among low achieving readers of either economic group. On another note, the types of uses of the computers were also related to income. Students in lower SES schools tended to use the computers for topics such as reading and mathematics versus students from higher SES who utilized the Internet on their computers (Judge et al, 2006). One does not know from the study what a self worth inventory or attributions might be, but it might be salient for further studies to investigate the relationship between FAE and technology education and performance in lower SES. 
The full force of lower SES students and their ability, or lack of ability, to be equipped for the future have considerable implications for our society; we need all of our citizens to perform optimally. If young students from lower SES are not afforded the full technology advances and opportunities of their upper income peers, then they may truly be at a disadvantage by the time they reach higher grade levels, such as high school. It is fundamental that we continue to lessen the impact of SES on technology integration in the classroom, but the solution is a global one and far beyond the reach of this paper. Social psychologists and educators will need to continue to work together to increase successful technology integration in the classrooms for students of all ages and all economic levels.  Ideally, through large-scale intervention and clearly defined research, lower SES students will be able to achieve their goals with technology as integral companion to their success. 
\section{Usability}
Learning and socialization in technology education is not necessarily equitable, and research demonstrates that socioeconomic differences are pronounced.  In their large longitudinal study on the “digital divide” between the haves and have-nots, Judge, Puckett, and Bell provide illuminating research (2006).  The authors clearly state their case for understanding the role of socioeconomic status (SES) and technology education:
If one assumes that academic achievement is facilitated by access to computers at 
home and at school, the gap in access to computer technology is cause for   
concern. Digital equity is a social justice goal, ensuring that all students have 
access to information and communication technologies for learning, regardless of 
socioeconomic status (SES), disability, language, race, gender... (Judge et al, 2006 
p.52).  
     The study sample included public school children that attended kindergarten, first and third grades.  Although Judge and her colleagues reported, from their findings, that overall access to computers and technology is increasing, they did find differences in technology access between high-and low-poverty schools (Judge et al, 2006).  Furthermore, their research demonstrated that high and low poverty schools still do not have enough classroom computers, indicating the need still exists for all children to have access to computers, despite SES. Truly, discrepancies do exist in technological availability and comprehension at school and at home, based on SES. The teachers’ training and expertise also affects student learning (Judge et al, 2006).  For social psychologists, this is also a significant issue, as those in authority do have the power and status to strongly influence outcomes.  Further research might yield greater understanding of the lower SES schools, teachers’ attitudes and technology outcomes for those students. Another study finding of Judge, and her colleagues, suggests the availability and location of the computers—whether in the classroom, readily available, or in a computer laboratory—influences use and obviously the children’s expertise in using the computers. Children will more readily use that which is easily accessible. Some could view such dynamics as environmental manipulation that precludes lower SES students from achieving their academic goals. Although this study didn’t research the specific issue of personal comfort and ease with using the computer, it could be theorized that increased familiarity would assist students in using the computer as a resource more readily and perhaps it could positively impact academic accomplishments. However, academic gains were not found with increased use of computers among low achieving readers of either economic group. On another note, the types of uses of the computers were also related to income. Students in lower SES schools tended to use the computers for topics such as reading and mathematics versus students from higher SES who utilized the Internet on their computers (Judge et al, 2006). One does not know from the study what a self worth inventory or attributions might be, but it might be salient for further studies to investigate the relationship between FAE and technology education and performance in lower SES. 
The full force of lower SES students and their ability, or lack of ability, to be equipped for the future have considerable implications for our society; we need all of our citizens to perform optimally. If young students from lower SES are not afforded the full technology advances and opportunities of their upper income peers, then they may truly be at a disadvantage by the time they reach higher grade levels, such as high school. It is fundamental that we continue to lessen the impact of SES on technology integration in the classroom, but the solution is a global one and far beyond the reach of this paper. Social psychologists and educators will need to continue to work together to increase successful technology integration in the classrooms for students of all ages and all economic levels.  Ideally, through large-scale intervention and clearly defined research, lower SES students will be able to achieve their goals with technology as integral companion to their success. 
Not all problems with technology are related to gender or economic biases.  There exist some problems inherent in the technology itself.  Of the articles referenced, one constant was explicitly stated: children learn computer technology more quickly than their adult counterparts.  However, as educators we must not overestimate students’ aptitude for technology.  Regardless of skill, computer use requires a level of cognitive attention that can detract from the intended lesson.  
Researchers Tancock and Segedy (2004) investigated such an issue in their classrooms. The authors described the obstacles implicit in technology in their experiment on reading comprehension in second grade classrooms.  Tancock’s and Segedy’s study sought, among other things, to discover how students comprehension and information recall would be affected by technology. Students were grouped according to three reading levels and were randomly assigned to either read the assignment on the computer or to read printed text—the latter being the control group. The students were asked to respond to particular activities using the computer or using paper and pencil (control).  Students completed self-evaluations on their performance and enjoyment of the assigned tasks.  The results showed that technology rich activities decreased reading comprehension and responses while, at the same time, students found the assignments more engaging and enjoyable.  Unexpectedly, the research found that students were naturally more collaborative when working in a technology rich environment (Tancock \& Segedy, 2004).  For social psychologists, this is an interesting phenomena, as there was an unexpected shift in group dynamics in their target group. This has tremendous applications for integrating technology into curriculum. Teaching the students on the computer in small groups might yield greater facility and comprehension on tasks on the computer.  Undirected internet use might be overwhelming for elementary students, since they often lack the mental focus to handle a non-linear source like a webpage or the ability to scan and skim—skills necessary to navigate the internet effectively (Tancock \& Segedy, 2004). The Tancock and Segedy study introduces several interesting concepts, primarily that effective computer education requires a certain level of mastery of the underlying technology to be most effective in learning and education. Attention to developmental issues, cognitive development and group dynamics could strengthen young students’ facility on the computers.  
Other investigators continued research that took the challenges of technology in education a step further (Levin \& Wadmany, 2006). In their study, the researchers sought to understand students’ views on learning with the information-rich tasks in a technology based environment and to glean understanding regarding the relationship between students’ perceptions of the use of technology versus their teachers educational beliefs. Additionally, unlike the majority of research that studied college students’ perceptions of learning and technology, this study worked with teachers and students in fourth thru sixth grade (Levin \& Wadmany, 2006). Their desire was to gain greater appreciation of the young students perspective on the use of technology in the classroom.  The researchers explored technology classroom uses in three categories: learning from, about and with Information and Communication Technology (ICT). (Levin \& Wadmany, 2006) Essentially, students and teachers were given open-ended questions and interviews regarding technological learning and perceptions.  The results showed that the students’ views are, not surprisingly, multidimensional, and the study identified various dimensions of learning. The concept known as explorative-thoughtful process was dominant in all the classrooms, at twice the frequency of two other dimensions (Levin \& Wadmany, 2006).  The findings revealed that ICT supports collaborative learning, which has direct applicability towards the solution for the integration of classroom technology (Levin \& Wadmany, 2006). The reciprocal and complex nature of the relationship between students’ and teachers was highlighted; it was found that teachers’ biases and views influenced their students on many levels (Levin \& Wadmany, 2006).  Consequently, the opportunity for teachers and students’ to learn from one another is invaluable; the innovations in technology education could ideally be the result of collaboration between students and teachers—a great solution for technology education.  The notion that technology amplified the insightfulness of young learners, and in fact, provided insight into themselves and their peers was an amazing find (Levin \& Wadmany, 2006). This study provides a springboard for further research for social psychologists to explore social cognitions, social influence and the interpersonal relationships to learning. The implications for forging new K-12 technology curriculum are endless and call for further research in this arena.

\chapter{Conclusion}
This paper has hopefully introduced the reader to several of the issues present in technology rich classrooms.  This paper aimed to identify and describe the problems of the integration of technology education in the classroom with a social psychological background and practice. Although it is impossible to provide immediate solutions to the problems, through the review of some pertinent studies, recommendations and best practices derived from research have been outlined.  By working with social psychologists, we, as educators, can improve the effectiveness of technology rich classrooms.  In the course of this paper, three specific problems have been highlighted: gender issues, socioeconomic concerns, and obstacles inherent in technology integration in the classroom. In review, this paper has also sought to provide three solutions as applications of the social psychology research.