\wrappedappendix{Special Notes for \LaTeX{} Users, Including a \\
	\hbox to 1.0in{}Demonstration of Wrapping Appendix Titles}
{Special Notes for \LaTeX{} Users, Including a Demonstration of Wrapping Appendix Titles}
%\chapter{Special Notes for \LaTeX{} Users}
\label{app:latex-notes}

\newcommand{\cmd}[1]{\texttt{$\backslash$#1}}

It is strongly recommended that you use this file as a template for your
thesis, since it greatly simplifies conforming to the required formatting
standards.

There are several important points that students using the \LaTeX{} version of
this template should verify before submitting a thesis.

\section{Front Matter}

Much of the front matter (i.e., the Roman numbered pages) is automatically
generated.  Use \cmd{renewcommand} command to customize the fields of these
templates.  For example,
\texttt{\cmd{renewcommand}$\{$\cmd{thesisauthor}$\}\{$your name here$\}$} will
customize the author name.

\sloppy
Most authors will need to customize the \cmd{thesismonth}, \cmd{thesisyear},
\cmd{thesisauthor}, \cmd{thesisauthorlastname}, \cmd{thesisdefensedate},
\cmd{thesistitle}, \cmd{thesisshorttitle}, \cmd{thesisdepartment},
\cmd{thesisfield}, \cmd{thesissupervisor}, and \cmd{thesiscommittee}
fields.  Examples of these can be seen in the sample \texttt{thesis-main.tex}
file.

\fussy
You must also specify \texttt{phdthesis}, \texttt{dscthesis}, or
\texttt{mastersthesis} when selecting the \cmd{documentclass}.  An
example can also be seen in the sample \texttt{thesis-main.tex} file.

\section{Table of Contents and Bibliography}

The Table of Contents is automatically generated.  \texttt{latex} should be run
twice in succession after making any changes to the Table of Contents.

Due to the way \LaTeX{} formats the Table of Contents, long appendix titles
will not automatically wrap and indent properly.  If you need to use a long
appendix title, you must manually wrap and indent the appendix's
table-of-contents entry.  The \cmd{wrappedappendix} command is defined in this
template to assist with this; an example is seen at the top of the sample
\texttt{thesis-appendixD.tex}.  This requirement only applies to appendix
titles: other section titles will automatically wrap properly, including
entries in the List of Tables and List of Figures.

If changes need to be made to the Table of Contents' formatting, you can use
the \cmd{addtocontents} command to insert some formatting commands
directly into the Table of Contents page.  More significant changes can be made
by editing the \texttt{.toc} file that \LaTeX{} automatically generates.
However, editing this file by hand is not recommended unless absolutely
necessary, since it will automatically be re-generated the next time \LaTeX{}
is run.

Like the Table of Contents, the Bibliography is automatically generated.  After
editing the bibliography file, you should run \texttt{latex}; run
\texttt{bibtex}; and re-run \texttt{latex} twice in succession.

\section{Widows and Page Breaks}

\LaTeX{} may create widows if you have a paragraph followed by a list.  To get
rid of this widow, you must force \LaTeX{} to break the page somewhere else.
Either insert a \cmd{newpage} command before the paragraph, or insert a
\cmd{samepage} command between the paragraph and the list.

\LaTeX{} may also create widows in the Tables of Contents.  You can force
\LaTeX{} to break the page in a more convenient location by inserting
\cmd{addtocontents$\{$toc$\}\{$\cmd{newpage}$\}$} before the corresponding
\cmd{chapter}, \cmd{section}, \cmd{subsection}, or \cmd{subsubsection} command
in the text.

Excluding these two situations, \LaTeX{} should not create orphans or widows.
However, in some situations it may place page breaks at strange places --- such
as several inches above the bottom margin --- in order to avoid creating
orphans or widows.  You can fix this by altering the \cmd{clubpenalty} or
\cmd{widowpenalty}, or by manually adding \cmd{newpage}s where \LaTeX{} guesses
incorrectly.
