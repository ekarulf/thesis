\chapter{Robotic Middleware}

A Body Area Network is formally defined by IEEE 802.15 as, a communication standard optimized for low power devices and operation on, in or around the human body (but not limited to humans) to serve a variety of applications including medical, consumer electronics / personal entertainment and other [IEEE 802.15]. In more common terms, a Body Area Network is a system of devices in close proximity to a persons body that cooperate for the benefit of the user. This paper discusses several uses of the BAN technology As IEEE mentioned, the most obvious application of a BAN is in the medical sector, however there are also more recreational uses to BANs. This paper will discuss the technologies surrounding BANs, as well as several common applications for BANs. At the end of the paper we will briefly discuss the challenges associated with BANs and some solutions that are on the horizon.

\section{Robot Operating System}
The field of computer science is constantly evolving to process larger data sets and maintain higher levels of connectivity. At same time, advances in miniaturization allow for increased mobility and accessibility. Body Area Networks represent the natural union between connectivity and miniaturization. A Body Area Network (BAN) is defined formally as a system of devices in close proximity to a persons body that cooperate for the benefit of the user. The BBC's Jo Twist gave a more informal definition of Body Area Networks in her article title When technology gets personal:

Inanimate objects will start to interact with us: we will be surrounded - on streets, in homes, in appliances, on our bodies and possibly in our heads - by things that think.

Forget local area networks - these will be body area networks.

Given the close proximity of users to the BAN technology, the demands on usability are exceptionally high. In section 3.3 we discussed Zheng et al. and the MobiHealth framework, we will again refer to the study as they represent some of the most advanced HCI design in the BAN field. Zheng et al. noted a usability problem with previous systems such as Lifeguard and AMON, the technology placed artificial restrictions on the user, which made adoption more difficult. Zheng's group decided to use advances in textile manufacturing to sensing wearable shirts that would actively monitor the wearer.[Zheng07] 


Interestingly enough, Zheng's group also found a usability fault in the EPI-MEDICS design, as the system would record ECG data and raise alarms as required, but it would only do so when requested by the patient. Zheng's group classified this as a usability flaw, as the usefulness of emergency detection sensors is in their detection of emergencies that are not planned. [Zheng07] 

Hopefully this paper introduced the reader to the BAN technology. We discussed the history and development of BSN and how that grew into the more general concept of BAN. We then introduced two refinements of the BSN concept, ABSN and MBSN. We discussed Human++ a ABSN flexible platform and the advantages and disadvantages with ABSN technology. We discussed MobiHealth as a mature example of MBSN technology. We then continued on to take a look at a case study involving MobiHealth and the monitoring cardiac data. We concluded the paper by looking at some challenges related to BAN. We covered signal and path loss in the human body and some of the challenges associated with communication and power within the human body. We covered usability and the fusion of  cutting edge technology and textiles and how it is shaping wearable technology. 

\section{Architecture}
Twist makes the possibility of BAN sound more like science fiction than a real possibility, but several experts in the field expect to see BAN in production for general use by 2010 [Schmidt02]. While this might seem like an aggressive estimate, when put into context with the history and development of BAN up to this point it becomes a much more achievable goal. In the paper we will start off introducing the reader to the history and development of BAN. We will cover the medical heritage of BAN and how the technology grew from a simple generalization of the concept of Body Sensor Networks (BSN). We will investigate current applications of BAN with an emphasis on applications in the medical sector. As we cover applications of BAN, we will spend a portion of the paper identifying some of technical problems facing BAN. Finally, we will conclude the paper with several solutions currently in development and how they hope to address and overcome the challenges inherent to BAN.

\subsection{Global Architecture}
BAN technology is still an emerging technology, and as such it has a very short history. BAN technology emerges as the natural byproduct of existing sensor network technology and biomedical engineering. Professor Guang-Zhong Yang was the first person to formally define the phrase Body Sensor Network (BSN) with publication of his book Body Sensor Networks in 2006. BSN technology represents the lower bound of power and bandwidth from the BAN use case scenarios. However, BAN technology is quite flexible and there are many potential uses for BAN technology in addition to BSNs. Some of the more common use cases for BAN technology are: Body Sensor Networks (BSN), Sports and Fitness Monitoring, Wireless Audio, Mobile Device Integration, Personal Video Devices.

Each of these use cases have unique requirements in terms of bandwidth, latency, power usage, and signal distance. IEEE 802.15 is the working group for Wireless Personal Area Networks (WPAN) [IEEE 802.15]. The WPAN working group realized the need for a standard for use with devices inside and around close proximity to the human body. IEEE 802.15 established Task Group 6 to develop the standards for BAN. The BAN task group has drafted a (private) standard that encompasses a large range of possible devices. In this way, the task group has given application and device developers the decision of how to balance data rate and power. Figure 1, below, describes the ideal position for BAN in the power vs data rate spectrum.

As you can see the range of BAN devices can vary greatly in terms of bandwidth and power consumption. The BAN draft requirements, displayed below, add a common set of requirements as to ensure that all devices conform to a similar set of behaviors yet still encompass a wide variety of devices as previously mentioned.

BAN technology is still emerging and there are a lot of problems left to solve. Setting aside ethical issues like privacy, there are still plenty of technical challenges that we must overcome before BAN will become an effective solution. The BAN draft submissions have defined solutions for a lot of the basic wireless network protocols, but there is still a large amount of research that must be done to effectively propagate a signal in and around the human body. The last challenge BAN technology faces is actually a problem of Human-Computer Interaction (HCI) and how to make the technology usable.


\subsection{Robot Architecture}
As previously mentioned, BANs have grown as a refinement of BSN. As such, BSN remain the most thought out applications of BAN. In his summary of the BAN task group's findings thus far, Stefan Drude, a researcher at Phillips, outlined the possible needs the group had found for the very low BSN devices. BSN devices refine the general requirements by restricting themselves to a much smaller range (0.01 - 2.00 m). This limited range allows developers to take advantage of several aspects of the human body. First, the human body itself can become a channel for short range communication, thus removing the need for a traditional antenna. By removing the requirement of an additional antenna, the power consumption of BSN devices shrinks to 0.1 - 1.0 mW. At this low of power, the human body is actually capable of generating enough excess energy that the devices could scavenge the required energy directly from the host's body, removing the restriction on traditional power sources (like batteries) [IEEE-BAN-SUMMARY]. BAN technology is not one that is unique to Mr. Drude and the members of the BAN task group, this exact use case scenario has been thoroughly described by Microsoft in their patent titled, Method and apparatus for transmitting power and data using the human body [Microsoft04]. In the following subsections, we will investigate systems that utilize the BSN technology to accomplish higher level tasks.

A managed body sensor network (MBSN) is defined as a system where the third party makes decisions based the data collected from one or many BSN. We will discuss MobiHealh and CodeBlue, two managed BSN that are are approaching development of managed BSN from two different perspectives.


In 2003, two researchers from the University of Twente published a paper entitled Continuous monitoring of vital constants for mobile users: the MobiHealth approach. The paper described the increasing demand of resources placed on the medical community, the rising costs of in-patient care, and the relative lack of out-patient monitoring. The paper defined extra-BAN communication (EBAN) as communication between a BAN and another network. The solution paper provided was MobiHealth, a BSN with EBAN connectivity to a 2.5/3G networks to provide out-patient monitoring of patients vital signs. Through this infrastructure the MobiHealth designers were able to provide sensor information to qualified medical professionals, where multiple patients data could be monitored in an aggregate form. [Konstantas03]

As one might expect, the signal and path loss inside the human body is drastically different than the rules in plain space. That said the rules governing signal and path loss remain the same. Researchers have been able to model signal loss throughout the human body, however the more interesting research involves using the human body as a transmission medium for electrical signals. Marc Wegmueller et al. have attempted to model the conductivity and permittivity of signals sent from one area of the body to another. A full summary of their research is beyond the scope of this paper, but it is worth noting that in the frequency range of 10 kHz to 1MHz, for every 5 cm between the transmitter and receiver there is an increase in attenuation by 6 to 9 dB. Other factors lowered or raised these constants, such as the geometry of the path, the amount of fat, and the presence of joints. [Wegmueller07]

\section{Panda3D Integration}
MobiHealth is simply one example of a managed BSN. Harvard University's Code Blue represents another example of BSN currently in the trial stages. Like MobiHealth, CodeBlue provides an infrastructure for multiple patient monitoring through EBAN communication. However, CodeBlue takes a more middleware approach to BSN instead of the packaged solution that MobiHealth provides. By providing a middleware layer, the CodeBlue project allows developers to specify the modules to use. In this way, CodeBlue is rather flexible at runtime. Two examples given by the MobiHealth team are emergency response and monitoring limb movement in stroke patient rehabilitation. Both scenarios have very different requirements both from a sensor perspective, and a timeliness perspective however the platform is able scale to accommodate both accordingly. [CodeBlue06]

Autonomous body sensor networks (ABSN) and MBSN share the same goals, but they accomplish them in different ways. While a MBSN will relies on reading sensor information and delivering it to a third party for decision making and intervention, ABSN take a more proactive approach. ABSN introduce actuators in addition to the sensors to allow the BSN to effect change on the users body. In addition to the actuators, ABSN contain more intelligent sensors that contain enough intelligence to complete their own tasks independently. [Gyselinckx05]

Human++ is a project developed in Belgium that aims to bring ABSN to the mainstream. The design of Human++ is relatively simple, any node in the mesh-network are able to talk to any other node in the network. There is a predefined central node that is designated for all EBAN communication. The central node also publishes information on any services that the ABSN provides external access to. An example ABSN diagram can be seen below in figure 2. [Gyselinckx05]

The most effective way of describing the current state of BSN is to actually describe a case study as a representative sample of the progress of BSN. In 2007 Zheng et al. published A wearable mobihealth care system supporting real-time diagnosis and alarm, a paper describing a MBSN using the MobiHealth infrastructure mentioned in the above section. We will briefly cover their design and implementation. This will help lead into our discussion of the challenges associated with BAN. [Zheng07]

The MobiHealth cardiac monitoring system implemented by Zheng et al. had one goal a few simple design principles and improvements over older MobiHealth products. The goal of the system was to provide long-term continuous monitoring of vital signs for high-risk cardiovascular patients. The project aimed for tight integration with GPS, which allowed system dispatchers to know the exact location of patients in distress. The project aimed to have a user friendly design that minimized the impact the monitoring system had on the patients. They accomplished this task using a Wearable Shirt comprised of smart fabric. The smart fabric was designed not only to provide sensor information wirelessly with the MBSN, but also to be resistant to casual wear and cleaning. The final system design was to provide online diagnosis and three separate levels of alarm on the local device. In this way, the design blended a little bit of ABSN technology into the system, by allowing the communication node to selectively raise events to dispatch only on anomalies, increasing the autonomy of the system. [Zheng07]
