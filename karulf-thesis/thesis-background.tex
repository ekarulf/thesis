\chapter{Exploration of Existing Interfaces}
\label{cpt:background}
\section{Robot User Interfaces}

Developmental psychology is the study of how people progress through life.  Until recently, most of developmental psychology focused on the development from birth to adolescence.  In recent years, developmental psychologists have begun the study of development at the other end of the age spectrum.   Older adults are commonly defined as over the age of sixty-five and the study of older adults is known as geriatrics.  After attending Dr. Brian Carpenter’s lecture on “Older patients and their physicians: getting and giving bad news,” I started to think about the other groups not represented by developmental psychology, young adults through middle-aged adults represent a majority of the population, however we have very little information on the age group.  Dr. Mitchell Sommers studies speech perception in older adults.  When asked about speech perception between the young adults in his study and the older adults in his study, Dr. Sommers responded, “we just don’t know how they get there.”  
I have chosen the study Efficacy of Donepezil on Maintenance of Activities of Daily Living in Patients with Moderate to Severe Alzheimer’s Disease and the Effect on Caregiver Burden to discuss as it addresses both the Alzheimer’s patients and their care givers. The drug Donepezil is a daily drug that is used primarily as a therapeutic treatment of Alzheimer’s disease, though there is no known study definitively proving a positive effect for treating Alzheimer’s, though smaller studies (such as the one in discussion) have shown small benefits vs. placebo.

\subsection{Visualization Interfaces}
The two hundred and ninety patients in this study were screened with the screening standardized Mini-Mental State Examination (sMMSE).  Patients scoring between five and seventeen were admitted into the study provided they maintained a level of “functional autonomy.”  In more common terms, participants were allowed into the study provided they were able to function independently in assisted care and community housing without the aid of dedicated 24x7 nursing staff.  In addition to the patients, their respective caregivers were also enrolled in the study.  An emphasis was placed on family and friends over paid caregivers where possible.  Caregivers were required to spend at least 8 hours with the patient at least 3 times a week, totaling a minimum of 24 hours of contact per week.  The study lasted for 24 weeks, and was conducted at several different geographic locations where the patients were randomly selected to receive either Donepezil or a placebo.  The patients, caregivers, and investigators were all not privy to which patients were enrolled in which group, this is known formally in psychology as a “double blind” study.  Patients in the experimental group were initially administered 5mg/day of Donepezil with the option of increase to 10mg/day per clinician’s judgment. 

\subsection{Control Interfaces}
The study measured outcome using two metrics, one for the patient and one for the caregiver.  The patients’ progress was measured using the 40-item Disability Assessment for Dementia (DAD) scale.  The DAD measures instrumental and basic activities of daily living (IADL and BADL respectively).  The caregivers’ progress was measured using the Caregiver Stress Scale (CSS).  This study is the first known use of the CSS within a clinical Alzheimer’s study, so slight modification was required.  A list of changed fields can be found on pg. 739.  The results were calculated by comparing both groups to the baseline using the least squares statistical model.  The results were modest, but positive for both patients and caregivers.  Patients rarely found clinical improvement, however patients in the experimental group found a smaller clinical decline in every category when compared to their counterparts in the control group.  Likewise, caregivers had few clinical improvements using the CSS, however the magnitude of decline was typically lower than the control group.  Finally, caregivers subjective ratings of the patient’s status showed caregiver’s in the experimental group on average felt their patient was doing significantly better than a similar pair in the control group. Finally, the study compared the amount of time a caregiver spent assisting the patient with BADL and IADL at weeks four, twelve, and twenty-four.  Caregivers in the experimental group typically spent less time assisting their patient when compared to the baseline, while caregivers in the control group typically spent more time assisting their patient when compared with to the baseline.


From these results we can conclude that Donepezil while it can not cure Alzheimer’s disease, it can at least slow the rate and magnitude of losses of ADL.  Caregivers of patients taking Donepezil spend less time caring for and assisting a patient with ADL which is positively correlated with decreasing caregiver stress.


\section{Video Game User Interfaces}
According to Renaud Donnedieu de Vabres, the cultural minister of France, “Video games are not a mere commercial product [ . . . ] they are a form of artistic expression involving creation from script writers, designers and directors” (qtd. in Crampton).  He is not alone in his opinion; prominent researchers at the University of Southern California express their agreement saying, “clearly, computer games are becoming a mainstream entertainment medium” (Jin, Lee, and Park 259).  This paper will clearly demonstrate that video games are a new and independent medium with a significant impact on our culture.	

\subsection{Real-Time Strategy}
In order to correctly describe a medium we must first agree on the terminology.  Since the concept of media has evolved over time, we must examine the description of media and sources in this book in terms of their historical context. In traditional terms, Webster’s Dictionary defines medium as “a means of effecting or conveying something” (“medium”). While there are many sources available on the topic of media, Kwan Mine Lee, Namkee Park, and Seung-A Jin from the University of Southern California currently have the most modern printed authority on the subject with their chapter Narrative and Interactivity in Computer Games published in 2006. They point out “one major characteristic of entertainment media is its focus on delivering a compelling narrative to the audience” (259).  For the purposes of this paper we will, like Jin, Lee, and Park, focus on entertainment media as a specific subset of media in general.
Three qualities are crucial to establishing videos games as entertainment media: narrative, interactivity and scope.  Narrative, the first essential element of entertainment media, dates back to stories told with cave walls paintings.  Because of this long history, one would assume that the definition of narrative is static but the concept has been actually quite volatile.  Aristotle revealed narrative in its most basic form by saying, “Narrative is a story that has a beginning, a middle, and an end” (Jin, Lee, and Park 264).  Skipping to the 20th century, the definition of narrative has evolved to, “an account of an agent whose character or destiny unfolds through actions and events in time” (qtd. in Jin, Lee, and Park 264).  In 2002, H.P. Abbott defined narrative as “the representation of a series of events” (Jin, Lee, and Park 264).  In truth, narrative is still developing.  
The second quality of entertainment media is exemplified by modern video games: interactivity.   More traditional forms of media have no precedent for describing narrative in terms of interactivity. Let us compare three video games as an example of the evolution of interactive narrative: DOOM, Myst, and World of Warcraft.  
DOOM is a heart-pounding epic action game as a US Marine attempts to fight his way out of hell.  The game is described as having a “linear” story line much like a book or a movie.  Jin, Lee, and Park describe linear narratives that, “always flow from the creator [ . . . ] to the audience,” where the extent of interactivity is mental and the audience can never physically reconstruct the plot (259).  DOOM fits in well with the early descriptions of narrative, as it accomplishes nothing differently from a narrative perspective than older media like books or movies. 
Myst is quite the opposite of DOOM from a content perspective.  A completely non-violent puzzle game that focuses on the character’s escape from an unknown island, Myst exemplifies what Jin, Park, and Lee describe as “narrative intensive” game (260).  The game’s puzzles and in turn the outcome depend the choices the player makes throughout the game, and unlike DOOM if played multiple times the player could change the course of the narrative.  This defiantly classifies the game as non-linear, however the game is also episodically reproducible.  Simply put, if a player made the same decisions and held the same behavior, the same outcome will always be rendered.  Myst has significantly fewer parallels in older media, however the strongest link to another medium lies with “create your own adventure” books that are popular with children.

\subsection{First-Person Shooter}
Our last case is the “Massively Multiplayer Online Role Playing Game” (MMORPG) World of Warcraft.  World of Warcraft is a continuation of a storyline from a previous game made by the same company.  In it, players are actually able to design their online avatar’s physical features, character type, and other details.  World of Warcraft takes interactivity to a new level with a plot line that provides loose boundaries for progression of the narrative and an infinite number of possibilities due to the freedom of choice of thousands of players in this virtual world.  Unlike the previous games, World of Warcraft has no counterpart in old media as the strong human interactivity creates infinite possibilities for a player’s narrative. 
The final concept that defines video games as an entertainment medium is that of scope.  Storytellers once relayed tales to small groups with oral discourse.  Later, handwritten and eventually printed narratives were circulated to the educated elite. In 1873 while working as a telegraph operator, Joesph May discovered that variations in light intensity could be transformed into electrical signals (Peters). It was not until 1925 in London, John Logie Baird, an electrical engineer from Scotland, demonstrated a machine that could put white letters on a black background from a distance. A decade later, the first broadcasts on this new medium were first seen publicly in 1935, in time for the 1936 Berlin Olympic Games (Peters). It was after 1936 that the popularity of televisions and its emergence as a medium of story telling truly began. Raymond Williams states the adoption of television impacted “not only on other entertainment and news media . . . but on some of the central processes of family, cultural [sic] and social life” (Williams and Williams 4).  Today we see the same impact of video games on a mass-market level.
Where are we now?  The adoption of video games has already begun. Riding on the successes of games in the early 1990’s, the video game industry has seen unprecedented success worldwide. Since 1996 the video game industry in the United States has grown from a \$2.6 billon industry, to a \$7.0 billion industry in 2005 (ESA 13). That same year, 2005, 69\% of American households played computer or video games (ESA 4). Even Time magazine weighed in on the popularity of video games saying video games have gone “from geek to chic in 33 years” (Grossman and Dell). This popularity is not specific to American culture; in January 2007 the makers of World of Warcraft announced that it had surpassed 8 million subscribers worldwide (Blizzard Entertainment, Inc.).  With all these successful titles and unique video games, “people start to wonder, Are [sic] video games art?” (Grossman and Dell). 
Ann-Sofie Axelsson and Tim Regan are both European researchers interested in online play. In their own words, the focus of their research is “describing and analyzing what people do, with whom they do it, and, perhaps most interesting, why they do what they do in online games” (291).  Their research shows that the player now shares a narrative with other players. This blurs the relationship between the creator and the participants, with participants independently interacting with other players. Their research also uncovered a key to the success of video games was the time spent “offline” or out of game.  A popular example of what Regan calls a “fan website” is Penny Arcade. 
Penny Arcade is a web comic about video games and the video game industry (Holkins and Krauhlik). This is in fact the very continuation of the communities created in the wake of id Software’s DOOM. Penny Arcade is an example of a successful web comic with 3.5 million readers; the site is a completely self-sustaining business (Khoo). The site is so successful, the site’s owners have setup an annual convention for video game players to meet up, play games, and see the video games still in development. Similar to independent film festivals in cinema, the convention – called Penny Arcade Expo – launched in the summer of 2004 with an attendance of 3,300 people over two days. The exposition has been hugely popular, growing to 19,323 attendees only two years later in 2006 (“Penny Arcade Expo”).  The growth of Penny Arcade Expo is a prime example of the growing impact of video games in modern culture. 
Noted critic and academic at Cambridge University Raymond Williams spent his life investigating modern culture. Described as “one of Britain's greatest post-war cultural historians,” Williams wrote several books on the media of television and cinema (Drummond).  Williams explains the advent of the television medium thusly: 
Television was invented as a result of scientific and technical research. As a powerful medium of communication and entertainment it took its place with other factors – such as greatly increased physical mobility, itself the result of newly invented technologies – in altering the scale and form of our societies. (Williams and Williams 3)
There are strong similarities between the development of television and the development of video games.  The idea of an interactive game was first described in a patent filed in 1948 by Thomas T. Goldsmith and Estle Ray Mann for a “Cathode-Ray Tube Amusement Device”. However the first documented video game would not be seen until a decade when young engineer by the name of William Higinbotham working at Brookhaven National Laboratory designed the game “Tennis for Two” to entertain visitors as they toured the laboratory (The First Video Game). In a small documentary on the history of Brookhaven, it was said that Higinbotham surprisingly chose not to patent his game as, “it didn't seem to be any more novel than the bouncing ball circuit in the instruction book” (The First Video Game).
