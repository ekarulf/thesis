%!TEX root = karulf-thesis.tex
\chapter{Introduction}

In his 1920 play \emph{R.U.R. ‘Rossum's Universal Robots’}, Karel Čapek coined the term ``robot'' from the Czech word \emph{robota} meaning ``forced labor.'' Čapek envisioned robots as biological machines, humans that were constructed to perform day-to-day tasks. Though the face of robots did not develop like Čapek predicted, his idea of robotics that serve individuals still holds promise. Steve Cousins, CEO of the robotics research institution \emph{Willow Garage}, suggests that ``in the next ten to twenty years our culture will experience a personal robotics revolution much like the personal computing revolution of the 1980's and 1990's''. \cite{Cousins} Dr. Cousins argues that in order to to achieve their goal of modern robotics we need to overcome difficult problems in hardware and software. This paper will address a subset of the software challenges that deal with the interactions between robots and humans.

Human-Robot Interaction, also known as HRI, is a discipline studying the interactions of humans and robots. This multidisciplinary field deals with draws on principles of artificial intelligence, social psychology, and Human-Computer Interaction. Human-Computer Interaction, know as HCI, is a well established field within the realm of Computer Science. The Association for Computing Machinery Special Interest Group on Human-Computer Interaction (SIGCHI) defines HCI as ``the study and practice of the design, implementation, use, and evaluation of interactive computing systems.'' \cite{SIGCHI} Due to the young age of HRI as a discipline, most of principles of HRI are adapted from the much broader field of HCI. Owing to these adaptations from HCI, existing HRI interfaces are limited to keyboard and mouse interfaces. While advances in hardware allow HCI scientists a much richer selection of interfaces, these interfaces have not become common place within the HRI community. A small selection of modern HRI interfaces are illustrated in Figure~\ref{existing-robot-ui}.

\begin{figure}[ht]
\begin{center}
\includegraphics[width=3.5in]{images/placeholder.png}
\caption{Existing 2D and 3D HRI interfaces\label{fig:existing-robot-ui}}
\end{center}
\end{figure}

Similarly, video games represent another young, multi-disciplinary field within the Computer Science community. Video games, like HRI, draw many of their principles from the world of HCI. Originally created as an application of computer graphics for use in the entertainment industry, video game development has matured into an independent industry in its own right. The popularity of video games has grown significantly throughout the past decade. The Entertainment Software Association estimates 68\% of American households now play computer or video games. \cite{ESA} This large demographic represents a pool of users already versed in exploring 3D virtual worlds interactively. In these interactions users are not only perceiving the a virtual world through simulated senses, but users are also performing actions and controlling their presence in this world through input devices, eg., a mouse and keyboard.

% TODO: Insert transition before "For example,"
Video games represent not only a large entertainment medium but they also have utility and direct applications to the fields of artificial intelligence and machine learning. Dr. Luis Von Ahn found that video games can be an effective tool for generating training data sets for machine learning. \cite{GWAP} Dr. Von Ahn developed special video games he calls ``games with a purpose''.  These games provide a way to solve problems that may be considered trivial for humans, yet are very challenging for computers. In these games the interaction of the human and the computer has changed to include the human in the computation of data.

Dr. Daniel Grollman, in his dissertation work, applied a video games interface with HRI principles to capture training data for robotic user interfaces. \cite{Grollman} Grollman's application, RGame, recorded user’s actions as they controlled a robotic dog and taught it to play soccer. Dr. Grollman was able to create an AI model for shooting and defending using the aggregate of data collected from RGame. Similar to Dr. Von Ahn's research, the RGame interface allowed the computer request information from the user to help it solve a problem.

As the field of robotics continues to evolve and mature, I predict the need for training data and effective user interface design will become more pronounced and prevalent. I propose that the robotics community should apply video game design concepts to robot control and display interfaces. I hypothesize that resulting interfaces will be intuitive and easy to use for users regardless of video game experience.

In order to defend this hypothesis I present the ``Robot Interactive Display Environment'', known as RIDE, as an example implementation of video game principles to address problems within the HRI space. I will examine prior work in the field of robotics user interfaces and present the strengths and weaknesses of the existing models. I will also identify several salient concepts from video game design and describe how these concepts improve the existing user interfaces. Finally, I will describe the testing procedures and results of a simple environmental search user study performed on the user interface.

