%%
%  This is all that frontmatter stuff
%
%  This way I can 'not' include it easily

% NOTE: do not put any text in the thesistitlepage, thesiscopyrightpage,
% or thesisdedicationpage sections.  If you want to use these pages, then you
% should remove the notes below (e.g., by uncommenting the \iffalse
% and \fi lines) and change the appropriate fields in thesis-main.tex.
% This will ensure that the copyright and dedication lines are positioned
% and formatted correctly.  Additionally, remove the
% thesisacknowledgmentpostscript and listoftablespostscript sections, since
% these are used to add explanatory notes which shouldn't be there in normal
% theses.

\begin{thesistitlepage}               %% Generate the title page.
\iffalse
\begin{singlespace}
\tiny
{\small \textbf{Important - How to use this document:}}
This sample document outlines guidelines for the proper formatting of theses
and dissertations for Master's and D.Sc.\ degree seeking students within the
School of Engineering at Washington University.  (Ph.D.\ students can also make
use of this document; see special note below.)  This document is formatted
using the same guidelines which it describes.  Consequently, by making an extra
copy of this document you can use it as a template into which you can insert
your own thesis or dissertation textual matter, replacing the original text
with your own while still retaining the general formatting contained within.
This document/template can be downloaded (as either a Microsoft WORD document
OR as a set of \LaTeX{} files) from the Engineering Student Services' website
and is located with other engineering graduate forms and guides.  Once
completed, hard copies of your document will be submitted for professional
binding, plus you will also need to submit your copy electronically, as per
procedures documented on the Engineering website's information for graduate
students.  Be certain to use your own full name wherever appropriate.  After
removing these comments, be sure to vertically center the information on this
title page to assure an equal amount of ``white space'' exists both above and
below your general block of title page information; however, be sure to leave
the vertical spacing between each area of text on the title page exactly as
shown above.  Your examination committee will likely contain only three members
if you are a Master's student, as shown in the sample above.  However, D.Sc.\
committees will typically have five members, and Ph.D.\ committees will have
six.  Make sure you use the month and year your degree is officially to be
\uline{earned} on the title page, abstract page, and on any vita page included.
\uline{If this is for your doctoral degree (i.e., either D.Sc.\ or Ph.D.)}, be
sure to change all occurrences of the word ``thesis'' to display as
``dissertation'', and change ``MASTER OF SCIENCE'' to ``DOCTOR OF SCIENCE'' or
DOCTOR OF PHILOSOPHY'', whichever applies.   \uline{IMPORTANT:  If you are a
Ph.D.\ student}, you must also change the line above (near the mid-section of
this page) to ``A dissertation presented to the Graduate School of Arts and
Sciences'' (but do NOT change the reference to the ``School of Engineering''
which is at the very top of this page, as that must be left exactly as shown.

{\small \textbf{Note for Ph.D.\ Students:}}
The formatting contained within this sample document can serve well in
emulating the basic formatting needed for the Ph.D.\ dissertation.  Be sure to
read ``Important reminders'' in paragraph above.\\ However, please remember
that all Ph.D.\ students are ultimately responsible for meeting the Graduate
School of Arts \& Sciences' formatting guidelines.  The GSAS dissertation
guidelines are published on the Graduate School website located with other
documentation for GSAS policies and guides.

\end{singlespace}
\fi
\end{thesistitlepage}

\begin{thesiscopyrightpage}                 %% Generate the copyright page.
\iffalse
\begin{singlespace}
\scriptsize
{\small \textbf{Important Notes Regarding Copyright Option:}} \\
Technically, a thesis or dissertation is protected to some degree by copyright
laws with or without a student having to register his or her claim to
copyright.  However, including a copyright page and applying for registration
of ones claim to copyright provide extra measures of legal protection from
potential copyright infringement.  There is a fee connected with explicitly
registering to copyright ones work; because of this, many students do not
choose to register to copyright their work.  Students should check with their
advisor(s) and/or seek legal advice to gather further information helpful to
making a decision with regards to registering their claim to copyright.  If you
are \uline{not} going to register to copyright your work, then you can choose
to remove this page from your document.  However, if you do choose to
explicitly copyright your work, then leave this page in, change the name to
your name, change the year to the appropriate year in which your degree will be
earned, and remove these notes of informational text.  If a student wishes to
officially ``register'' this claim to copyright, then Masters students will
need to pursue that effort on their own and can find appropriate options by
searching the web.  Doctoral students can complete an authorization to apply
for registration (i.e., of their claim to copyright the dissertation) by
indicating this interest in the appropriate area on the online form when
submitting the final electronic copy as per procedures documented on the
Engineering website's information for graduate students.

{\small \textbf{Important Notes Regarding Page Numbering and Margins:}} \\
If you decide to include this copyright page in your final document, do
\uline{not} count the page among your counted pages, and do \uline{not} display
any page number on the page.  \uline{Every sheet of paper in the manuscript
should be numbered except for two:  the title page and this optional copyright
page.}  Specifically, the front textual information (which comes before your
main thesis/dissertation body of text) is numbered with Roman numerals, and
your main body of text begins with Arabic numbers.  Since the title page is
counted but \uline{not} numbered, roman numeral \uline{``ii'' is always the
first number used and appears on the page AFTER the title page (AND AFTER the
copyright page, IF included)} --- as shown in this sample template document.
Page numerals should always display centered, just above the 1 bottom margin.
The left margin should be 1.5 inches, with a 1 inch margin at top, bottom, and
right.  The left margin is extra-wide in order to accommodate the binding
process.  When typing the manuscript, stay well within these margin guides.
Lastly, remember to update your table of contents such that the page numerals
referenced there will match the page numbers on the bottom of the pages to
which they make reference in your document.  This is necessary to do manually
because, unfortunately, the page numbering within this templates table of
contents is \uline{not} automatically linked to the pages of the body of text.
This is further documented, along with some work arounds, in the appendix to
this guide called Special Notes for MS WORD Users.  \LaTeX{} users may have to
invent other solutions with regards to synchronizing table of contents page
references with actual document page numbers.  This guide merely provides a
helpful starting point.  \textbf{REMINDER:} When you remove these comments, be
sure to leave the copyright information centered both vertically and
horizontally on the page (if you decide to explicitly copyright your work).
\end{singlespace}
\fi
\end{thesiscopyrightpage}

\begin{thesisabstract}
% TODO: Write Abstract
Sample text Sample text Sample text Sample text Sample text Sample text
Sample text Sample text Sample text Sample text Sample text Sample text
Sample text Sample text Sample text Sample text Sample text Sample text
Sample text Sample text Sample text Sample text Sample text Sample text
Sample text Sample text Sample text Sample text Sample text Sample text
Sample text Sample text Sample text Sample text Sample text Sample text
Sample text Sample text Sample text Sample text Sample text Sample text
Sample text Sample text Sample text Sample text Sample text Sample text
Sample text Sample text Sample text Sample text Sample text Sample text
Sample text Sample text Sample text Sample text Sample text Sample text
Sample text Sample text Sample text Sample text Sample text Sample text
Sample text Sample text Sample text Sample text Sample text Sample text
Sample text Sample text Sample text Sample text Sample text Sample text
Sample text Sample text Sample text Sample text Sample text Sample text
Sample text Sample text Sample text Sample text Sample text Sample text
Sample text Sample text Sample text Sample text Sample text Sample text
Sample text Sample text Sample text Sample text Sample text Sample text
Sample text Sample text Sample text Sample text Sample text Sample text
Sample text Sample text Sample text Sample text Sample text Sample text
Sample text Sample text Sample text Sample text Sample text Sample text
\end{thesisabstract}

\iffalse
\renewcommand{\thesisacknowledgmentpostscript}{
\textbf{Reminders of what needs to be updated:}
After removing these comments, use the above format to help input your
acknowledgments page.   A special dedication can be placed as the final
paragraph, as shown above; alternatively, you may include a special dedication
on the page that follows, as also shown in this sample template.}
\fi

\begin{thesisacknowledgments}
I would like to acknowledge the National Science Foundation for their financial
support of this research. I would also like to thank Willow Garage for their
financial and engineering support.

A special thanks goes to Parker Dunton, Marshall Strother, and Dan Lazewatsky
for their help with the development of RIDE.

Finally, I would like to thank the Washington University community for showing
me how short six years can be.
\end{thesisacknowledgments}

\begin{thesisdedicationpage}                %% Generate the dedication page.
\iffalse
\textbf{Note:} You may include a special dedication as shown here.  If you
include this page, be sure to keep it brief and center it on the page both
horizontally and vertically.  Alternatively, you may remove this page
altogether, and a special dedication can be placed as the final paragraph to
your acknowledgments page (as shown in this document on the preceding page).
\fi
\end{thesisdedicationpage}

\begin{singlespace}
\tableofcontents

\iffalse
\renewcommand{\listoftablespostscript}{
\small
\textbf{Note:} Be consistent in aligning multi-lined table-names, figure-names,
and chapter/section-names throughout your document.  It is generally
recommended to make sure any additional lines (i.e., within a long title or a
long table name) wrap and align immediately under the 1st character of the
title or name with which they are associated in the line immediately above ---
as shown in the ``Table 2.1'' example above.   Whatever approach you take, be
consistent.}
\fi

\listoftables

\listoffigures
\end{singlespace}

	
%%
%% For List of Abbreviations, Glossary or Nomenclature also
%% use \chapter, but put some kind of list environment inside.



%%% Local Variables: 
%%% mode: latex
%%% TeX-master: "thesis-main"
%%% End: 
