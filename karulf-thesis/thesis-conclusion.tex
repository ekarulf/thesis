%!TEX root = karulf-thesis.tex
\chapter{Conclusion}

We propose applying techniques and design elements from video games to address weaknesses in existing human-robot interfaces. We submitted RIDE, an interface for controlling heterogeneous robot teams using design elements from video games. We introduced the concept of ``sliding autonomy'' to allow the human operator and the robot to vary the degree of robot autonomy dependent on the situation. We presented an overview of the implemented system, RIDE, and using the ROS middleware framework. Finally we present the results of preliminary user studies showing, among other results, the effectiveness of our notification system.

\section{Discussion} % (fold)
\label{sec:discussion}

The results presented in Section~\ref{sec:study-results} show, unsurprisingly, that subjects who play video games or have prior experience controlling robots are better at performing the search task (ie. have lower completion, total neglect, and total idle times) with the RIDE interface than those who do not regularly play video games. While experience with RTS games seems to have no significant effect on the how well subjects perform the task, experience with first-person games clearly demonstrates an effect, decreasing all three times. First-person game experience caused subjects to use the first-person interface less than subjects who did not play first-person games, suggesting experienced first-person gamers recognize limitations of the interface.

All subjects, regardless of experience, showed a preference towards the supervisory mode interface spending over 69\% of their time, on average, in this mode. The first-person mode was only used by a single subject during non-training runs. This leads us to believe limitations in our current implementation of the first-person are preventing adoption. Currently, the map is rendered as a texture on the ground plane in all modes, making it hard to see and correctly interpret at acute viewing angles. We hypothesize extruding the map into 3D space, effectively creating ``walls'' coming out of the floor, would provide a first-person user a much better sense of positional awareness with respect to the map. The idea of extruding the map to create 3D walls is similar to an interface by Bruemmer et al \cite{Bruemmer05turnoff}.

The percentage of time spent in supervisory mode dramatically affects the times, with higher percentages correlating with much faster completion, and much reduced neglect and idle times. This suggests that an RTS interface is more appropriate than first- or third-person interfaces for this type of search task.

Notifications have a similarly beneficial effect, making the task subjectively easier to perform, and causing the subjects to ask for help less frequently. Surprisingly, across our whole population, notifications did not have significant effect on completion time, total neglect time, or total idle time.

However, the use of notifications did have a significant effect on subjects with no prior robot control experience, lowering completion time, neglect time, and idle time dramatically. We attribute this difference in effect to a flaw in our experimental design. We believe that the two-robot search task was not taxing enough for those who had controlled robots before. We hypothesize that these advanced subjects were able to manually monitor both robots at the same time, actively anticipating failures, making the notification system redundant. More inexperienced users, on the other hand, might not notice a robot in need of help when attending to the other robot, leading to increased total neglect and idle times. In this situation, notifications serve a critical function.

In conclusion, we believe that mixed-mode video-game-based interfaces, such as RIDE, offer an effective way to control large numbers of mostly-autonomous mobile robots. Our initial studies suggest that an active notification system allows inexperienced users to more efficiently control teams of robots, reducing neglect and idle times. We believe that this result will also apply to more experienced users in situations where they are sufficiently overloaded, or where unexpected events happen frequently.

% section discussion (end)

\section{Future Work}
\label{section:futurework}
Observing the user interface usage, during the user studies, revealed several weak points in our implementation. We believe that small enhancements to RIDE will greatly benefit the end user experience.

In Section~\ref{sub:info_exchange} we introduced Nielsen's prior work designing ecological interfaces for robotics. While the paper was a large influence on our design work, we feel our interface would benefit from additional visual cues. In their proposed UI, Nielsen extruded the known elements from the coverage map vertically to create an approximate three dimensional representation \cite{Nielsen_Teleoperation}. In our original design, we rejected this design element as we believed it misrepresented the third dimension to the user. However, during user studies, the users often mentioned it was difficult to position the camera for teleoperation while keeping the map visible.

During our user studies we also discovered that many users, especially users without prior robotics experience, found the robots to behave in a counterintuitive manner. As an example, the path planning algorithm we used in our study is weighted in a way that often chooses a longer, more circuitous route to avoid obstacles, than a more direct route that passes near obstacles. We believe that exposing more behavioral information will reduce the cognitive load on users. Future versions of the RIDE application would benefit from adapting the prior work in visualizing autonomous navigation systems to the field of robotics.

\section{Future User Studies}
\label{sec:futurestudy}
While successful, the design of the preliminary user study introduced several mitigating factors that prevented a definitive acceptance of our hypothesis. We feel our hypothesis can be clearly demonstrated with a revised experimental design. We identify and propose solutions to address several major problems with our experimental design.

In our preliminary user study, our intent was to provide a common set of knowledge in terms of robotics and video game experience. Unfortunately, by providing this training we made the analysis between experience groups much more difficult. As a result, we have widened our target audience to include users without robotics experience. In future studies, we plan to limit our training to include only speak-aloud practice to avoid introducing any potential biases.

In addition to the problems in our pre-experiment preparation, we have identified several ways to improve upon the experimental design itself. The purpose of our experiments was to prove applying video game principles to HRI interfaces will improve usability. In order to prove our hypothesis, we used notification display as an independent variable and recorded completion time as a measure of usability. While this is a valid approach, we would like to re-focus our experiment to directly test our mixed-mode interface by measuring cognitive load. 

In order to evaluate the sliding autonomy approach, we will run a randomized three-condition within subject experiment. Instead of using notifications we will use the control modes as our independent variable. This gives us three conditions to test: direct only control, supervisory only control, and blended control. This would allow us to directly compare the given modes and record metrics on the blended control interfaces, such as time spent in each mode and the number of mode switches.

As mentioned above, while we can correlate successful task completion with improved usability, we would like to utilize a more direct metric. Researchers at the United States National Aeronautics and Space Administration (NASA) developed a metric for describing a human's workload while accomplishing a task. The metric, known as the ``Task Load Index,'' is computed from a user's perceived workload on six sub-scales: mental demand, physical demand, temporal demand, performance, effort, and frustration \cite{NASA_TLX}. Our revised user study would ask subjects to rate their workload on each of the Task Load Index's six sub-scales after each experiment. At the conclusion of the user study, we would ask users to answer the fifteen questions to determine the weighting of the sub-scales when computing the task load index \cite{NASA_TLX20}.

Given the context of testing the control modes using the NASA Task Load Index, we decided to modify the subjects' tasks to more accurately represent our use cases. These new - more challenging - tasks would require a larger simulated world, more robots, and a different scenario. A weakness in the preliminary user study's scenario, is that the robots were able to complete their task immediately. This model does not match a use case in the real world, where most actions will require a non-zero amount of time to complete. Such a study, we believe, will show the benefits of mixed-mode interfaces, across the whole population, when even experienced users cannot realistically monitor all robots at once.

