%!TEX root = karulf-thesis.tex
\chapter{Introduction}
Steve Cousins, CEO of the robotics research institution \emph{Willow Garage}, suggests that ``in the next ten to twenty years our culture will experience a personal robotics revolution much like the personal computing revolution of the 1980's and 1990's.'' \cite{Cousins} While we agree with Dr. Cousins, we believe the advancement of robotics in our society is limited by control interfaces. The emphasis of robot control software focuses on teleoperation of a single robot by a single human. In this paper, we explore the limitations of existing robot human-robot interfaces, and we present a new paradigm for human-robot interaction.

We believe there is a growing need for robot control interfaces that allows a single user to control multiple robots simultaneously. In Section~\ref{sub:hri_prior_work}, we describe several existing interfaces designed for a single human to control one robot. We believe these existing interfaces will not scale to support more complex, real-world tasks. We propose a paradigm for human-robot interaction known as sliding autonomy. Sliding autonomy grants the human operator a high-level abstraction for providing tasks to a robot's artificial intelligence while retaining the flexibility to directly teleoperate a robot. This dynamic level of autonomy grants users the ability to control multiple robots concurrently. While the concept of sliding autonomy is new to human-robot interaction, it is well established within other disciplines.

We argue that interfaces, based on elements from computer video games, are effective tools for the control of large robot teams. In Section~\ref{sec:video_game_interfaces} we introduce several genres of video games. We describe how to apply elements from these various video game genres in robot control interfaces. We present RIDE, the Robot Interactive Display Environment, as an example of such an interface. RIDE contains two different modes: a supervisory control mode and a direct control mode; these modes represent the two extremes of autonomy. The supervisory mode borrows display and control mechanics from real-time strategy games to allow task-oriented control of many robots. The direct mode borrows visual elements from racing games for a more fine-grained control of a single robot. Additional details of RIDE's design - including the transition between the two modes - is described in Section~\ref{sec:ride_user_interface}.

We conducted a formal user study to evaluate the effectiveness of the RIDE interface, and specifically the notification system. Our two-condition experiment asked subjects to perform a search task using two simulated robots in a small house. We instrumented the application to record the user's behavior and timing information. The results of this data, combined with the pre-experiment and post-experiment questionnaires, are presented in Section~\ref{sec:study-results}.