%!TEX root = karulf-thesis.tex
\chapter{Introduction}
Steve Cousins, CEO of the robotics company \emph{Willow Garage}, suggests, ``in the next ten to twenty years our culture will experience a personal robotics revolution much like the personal computing revolution of the 1980's and 1990's'' \cite{Cousins}. While we agree with Dr. Cousins, we believe the advancement of robotics in our society is limited by control interfaces. There is a growing need for robot control interfaces that enable a single user to effectively control more than one remote robot \cite{RIDE_HRI11}. The increasing levels of autonomy demonstrated by our robots allow them to be controlled at progressively higher levels of abstraction. Such systems are not perfect; human intervention is frequently required when unanticipated circumstances develop, or an object beyond the capability of the perception systems must be evaluated \cite{RIDE_HRI11}.

The requirement for intermittent intervention suggests that our interfaces should be capable of both high-level (task-based) and low-level (direct teleoperation) control of remote robots, as well as the capacity to easily switch between these modes, as necessitated by the situation \cite{RIDE_HRI11}. The goal of a single operator controlling many robots further indicates the need for a system which allows an individual to alert the operator when help is needed. If there are too many robots for the operator to attend to at once, there is a clear possibility a robot may sit idle, waiting for assistance, for an extended period of time, without such a notification system \cite{RIDE_HRI11}.

Many of the obstacles of controlling remote robots are similar to the challenges of controlling characters in video games. Real-time strategy games require controlling many tens, or hundreds, of heterogeneous units at once. First-and third-person games involve the detailed direct control of a single character. We believe that the interfaces utilized for these games make ideal candidates for interfaces for mobile robot control \cite{RIDE_HRI11}.

Our motivation for drawing from video games is straightforward. Video games with easy-to-use, effective interfaces will be played more often, and they will sell better than games with poor interfaces. Almost four decades of market forces have refined interfaces for many genres of games; this has resulted in systems that are intuitive, easy to use, and effective. Furthermore, since many people play these games, they are already familiar with the interfaces. We hypothesize that these facts will make robot control interfaces based on computer game interfaces highly effective tools \cite{RIDE_HRI11}.


% The emphasis of robot control software focuses on teleoperation of a single robot by a single human. In this paper, we explore the limitations of existing robot human-robot interfaces, and we present a new paradigm for human-robot interaction. We believe there is a growing need for robot control interfaces that allows a single user to control multiple robots simultaneously. In Section~\ref{sub:hri_prior_work}, we describe several existing interfaces designed for a single human to control one robot. We believe these existing interfaces will not scale to support more complex, real-world tasks. 


% We propose an existing paradigm of human-robot interaction known as sliding autonomy. Sliding autonomy grants the human operator a high-level abstraction for providing tasks to a robot's artificial intelligence while retaining the flexibility to directly teleoperate a robot. This dynamic level of autonomy grants users the ability to control multiple robots concurrently. While the concept of sliding autonomy is new to human-robot interaction, it is well established within other disciplines.

% We argue that interfaces, based on elements from computer video games, are effective tools for the control of large robot teams. In Section~\ref{sec:video_game_interfaces} we introduce several genres of video games. We describe how to apply elements from these various video game genres in robot control interfaces. 

We present RIDE, the Robot Interactive Display Environment, a control interface for robots that draws heavily on computer game interfaces for inspiration. RIDE combines aspects from a number of computer game genres, allowing the operator to switch between direct and supervisory control, as the situation requires. Unlike existing interfaces in literature, RIDE allows both the user and the robot to negotiate the level of autonomy. RIDE achieves this negotiable level of autonomy by integrating several types of interfaces: a supervisory control mode and a direct control mode. The supervisory mode borrows display and control elements from real-time strategy games to allow task-oriented control of many robots. The direct mode borrows visual elements from racing games for a more fine-grained control of a single robot. Additional details of RIDE’s design—including the transition between the two modes—is described in Section~\ref{sec:ride_user_interface}.

To evaluate the effectiveness of the RIDE interface, specifically the notification system, we conducted a formal user study. Our experiment asked subjects to perform a search task using two simulated robots in a small house. The experiment had two conditions: one with notifications displayed, and one with notifications hidden. We instrumented the application to record the user’s behavior and timing information. The results of this data, along with the pre-experiment and post-experiment questionnaires, are presented in Section~\ref{sec:study-results}. In addition to discussing video game interfaces and presenting RIDE, we present the results of our initial user studies with the interface, which suggests it is well suited to search tasks with multiple robots.
